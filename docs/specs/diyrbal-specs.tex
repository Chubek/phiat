\DocumentMetadata{}
\documentclass[a4paper,12pt]{article}
\usepackage{xspace}


\newcommand{\nm}{Diyrbal\xspace}
\newcommand{\gcmeth}{Mark\&Compact\xspace}

\begin{document}

\section{General description \& features}

\subsection{\nm: A short introduction}

\nm is a general-purpose, extensible, type-safe and statically-typed imperative Wirthian-esque scripting/programming language with stellar support for multi-paradigm programming (functional and object-oriented). It is both AoT'd\footnote{Ahead-of-time compilation} and JIT'd\footnote{Just-in-time compilation}.

Furthermore, \nm uses constraint-based type inference (a feature of functional languages, and more so, Rust) to allow user to forgo specification of types during programming.

After writing your scripts and programs, you specify whether you wish for your program to be interpreted on-the-go, or compiled to a process image, which is later read by a VM. The on-the-go interpretation also relies on a VM, but the set of opcodes used for interpretation, and compilation, are different.

\nm ships with a runtime to execute the process images. The users can download, and install the runtime only. You may have noticed a lot of similarities with Java now. In a way, \nm is a hybrid of Java, and the less byzantine scripting languages of Unix (Perl, Python, Ruby).

Both the VMs use \gcmeth method of garbage collection. % TODO: add more info on GC

In the following sections, we'll explain the features yieled by this implementation of \nm.

\subsection{Functional features}



\end{document}
